\documentclass[11pt,a4paper]{article}

\usepackage{epsfig}
\usepackage{multicol}

\usepackage[utf8]{inputenc}
\usepackage[brazil]{babel}
\usepackage{fancyheadings}
\usepackage{amsmath}
\usepackage{calrsfs}
\usepackage{enumerate}
\usepackage{enumitem}   
\DeclareGraphicsExtensions{.png,.pdf}
\usepackage{amsmath, amsfonts, amssymb}
\usepackage{esint}
\usepackage{graphicx}
\usepackage{multicol}
\usepackage{tasks}
\usepackage[utf8]{inputenc}
\usepackage{mathrsfs} % Transformada de Laplace
\usepackage{indentfirst}
\usepackage{xcolor}
\usepackage{pgfplots}   % pacote para uso do pgfplots

% As margens
\setlength{\textheight}{24.0cm}
\setlength{\textwidth}{17.5cm}
\setlength{\oddsidemargin}{2.0cm} % Margens reais desejadas
\setlength{\evensidemargin}{2.0cm} % 2+17.5+1.5=21cm (largura A4)
\setlength{\topmargin}{1.5cm} % 1.5+1.6+1.0+24.0+1.6=29.7cm
\setlength{\headheight}{1.6cm} % (altura A4)
\setlength{\headsep}{1.0cm}
\setlength{\columnsep}{1.5cm} % Coluna = 8cm ((17.5-1.5)/2)
\addtolength{\oddsidemargin}{-1in}
\addtolength{\evensidemargin}{-1in}
\addtolength{\topmargin}{-1in}
\setlength{\footskip}{0.0cm}


% Novos comandos
\newcommand{\limite}{\displaystyle\lim}
\newcommand{\integral}{\displaystyle\int}
\newcommand{\somatorio}{\displaystyle\sum}
\newcommand{\mat}[1]{\mbox{\boldmath{$#1$}}} 

\pagestyle{fancy}


\usepackage{lipsum}

\lhead{
\includegraphics[width=1cm]{brasao.png}
}

\rhead{ 
\sc\textbf{U}niversidade \textbf{F}ederal do \textbf{C}eará\\
Campus Quixadá\\ Inteligência Artificial}

\cfoot{}

\begin{document}

	\begin{center}
		\Large Resolvedores SAT - O Problema das N Rainhas  
	\end{center}

\begin{flushleft}
\textbf{Nome:} Mateus Sousa Araújo \\
\textbf{Matrícula:} 374858 \\
\textbf{Professora:} Maria Viviane de Menezes \\
\textbf{Curso:} Engenharia de Computação \\
\end{flushleft}

\begin{enumerate}

\item \textbf{Realização de Testes}

O código foi feito em C++ e foram realizados 10 testes com a ordem do tabuleiro e o tempo de execução em segundos.

% ######## init table ########
\begin{table}[h]
 \centering
% distancia entre a linha e o texto
 {\renewcommand\arraystretch{1.25}
 
 \begin{tabular}{ l l }
  \cline{1-1}\cline{2-2}  
    \multicolumn{1}{|p{3.850cm}|}{\textbf{Ordem N} \centering } &
    \multicolumn{1}{p{4.217cm}|}{\textbf{Tempo (s)} \centering }
  \\  
  \cline{1-1}\cline{2-2}  
    \multicolumn{1}{|p{3.850cm}|}{10 \centering } &
    \multicolumn{1}{p{4.217cm}|}{0.010 \centering }
  \\  
  \cline{1-1}\cline{2-2}  
    \multicolumn{1}{|p{3.850cm}|}{20 \centering } &
    \multicolumn{1}{p{4.217cm}|}{0.049 \centering}
  \\  
  \cline{1-1}\cline{2-2}  
    \multicolumn{1}{|p{3.850cm}|}{30 \centering } &
    \multicolumn{1}{p{4.217cm}|}{0.073 \centering}
  \\  
  \cline{1-1}\cline{2-2}  
    \multicolumn{1}{|p{3.850cm}|}{40 \centering } &
    \multicolumn{1}{p{4.217cm}|}{0.223 \centering}
  \\  
  \cline{1-1}\cline{2-2}  
    \multicolumn{1}{|p{3.850cm}|}{50 \centering } &
    \multicolumn{1}{p{4.217cm}|}{0.501 \centering}
  \\  
  \cline{1-1}\cline{2-2}  
    \multicolumn{1}{|p{3.850cm}|}{100 \centering } &
    \multicolumn{1}{p{4.217cm}|}{7.826 \centering}
  \\  
  \cline{1-1}\cline{2-2}  
    \multicolumn{1}{|p{3.850cm}|}{120 \centering } &
    \multicolumn{1}{p{4.217cm}|}{17.754 \centering}
  \\  
  \cline{1-1}\cline{2-2}  
    \multicolumn{1}{|p{3.850cm}|}{150 \centering } &
    \multicolumn{1}{p{4.217cm}|}{57.178 \centering}
  \\  
  \cline{1-1}\cline{2-2}  
    \multicolumn{1}{|p{3.850cm}|}{190 \centering } &
    \multicolumn{1}{p{4.217cm}|}{240.09 \centering}
  \\  
  \cline{1-1}\cline{2-2}  
    \multicolumn{1}{|p{3.850cm}|}{200 \centering } &
    \multicolumn{1}{p{4.217cm}|}{300.559 \centering}
  \\  
  \hline

 \end{tabular} }
 \caption{Testes realizados}
\end{table}

\item \textbf{Geração do gráfico}

Logo após gerei um gráfico com a quantidade de rainhas e o tempo de execução. Foi observado um crescimento exponencial de acordo com o aumento do número de rainhas. Se N for muito grande, o tempo de execução também é muito grande, necessitando dessa forma, de altos recursos computacionais. O número de cláusulas cresce exponencialmente. Podemos considerar esse problema como NP-Completo.

{\par\centering
\begin{tikzpicture} %  indicação que estamos iniciando uma figura

\begin{axis}[ %  iniciando os eixos   
xlabel= Número de rainhas (N),            % nomeando os eixos
ylabel= Tempo (s)]

\addplot[color=blue,mark=*] coordinates{  % caracterizando a plotagem
(10,0.010)
(20,0.049)
(30,0.073)
(40,0.223)
(50,0.501)
(100,7.826)
(120,17.754)
(150,57.178)
(190,240.09)
(200,300.559)
};
\end{axis}

 
\end{tikzpicture}

\par}

\end{enumerate}


\end{document}
